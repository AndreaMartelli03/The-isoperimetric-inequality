\begin{lemma}[Constrained isoperimetric inequality]\label{lemma: constrained isoperimetric inequality}
    Let \(m,R>0\) with \(m<\omega_nR^n\). Then 
    \[
        \alpha(m,R) \coloneq \inf\{ P(E) \colon E \subset B_R \text{ measurable, } |E|=m\}
    \]
    is realized, and every minimizer is equivalent to a ball. In particular, \(\alpha(m,R)\) is the perimeter of a ball of volume \(m\), that is
    \[
        \alpha(m,R) = n \omega_n^{1/n} m^{\frac{n-1}{n}}.
    \]
\end{lemma}
\begin{proof}
    That \(\alpha(m,R)\) is realized is a consequence of the Direct Method, as bounded sets in \(BV(B_R)\) are precompact with respect to the BV-strict convergence, the family of characteristic functions of sets in \(B_R\) is closed, and any limit in the class preserves the volume. 

    Let us now assume that \(E \subset B_R\) satisfies \(|E|=m\) and \(P(E)=\alpha(m,R)\). Since \(|E\triangle E^{(1)}|=0\), we may assume that \(E=E^{(1)}\). For every unit vector \(v \in \R^n\), \(|v|=1\), by Theorem~\ref{thm: steiner inequality} we have that \(P(S_v(E))\le P(E)\). On the other hand, by the properties of the Steiner symmetrization \(|S_v(E)|=m\) and \(\diam(S_v(E)) \le \diam E<R\), so \(P(S_v(E)) \ge \alpha(m,R)=P(E)\). Then by Theorem~\ref{thm: steiner inequality} and Lemma~\ref{lemma: density points and vertical sections} 
    \[
        E_{v,z} \coloneq \{ s \in \R \colon z + sv \in E\}
    \]
    is an interval for every \(z \in v^\perp\). 

    We claim that \(E\) is convex. Indeed, let \(x,y \in E\) be two distinct points, and set \(v = (y-x)/|y-x|\). Then if \(z \in v^\perp\) is the projection of \(x\) (and \(y\)) in \(v^\perp\), 
    \[
        (1-\lambda)x + \lambda y = z + [(1-\lambda)x\cdot v + \lambda x \cdot v] v \in E, \qquad \forall \lambda \in [0,1]
    \]
    as \([x\cdot v, y\cdot v] \subset E_{v,z}\).

    By Theorem~\ref{thm: steiner inequality}, for every unit vector \(v \in \R^n\) there exists \(c_v \in \R\) such that \(E= S_v(E) + c_v v\). Consider
    \[
        E' \coloneq E - (c_{e_1}e_1 + \cdots c_{e_n}e_n).
    \]
    Since \(E'\) is still convex and for every unit vector \(v \in \R^n\) we still have \(P(E')=P(S_v(E'))\), there are constants \(c_v' \in \R\) such that \(E' = S_v(E')+c_v'v\). But by construction, 
    \[
        S_{e_i}(E') = S_{e_i}(E) - \sum_{j \ne i} c_j e_j = E-c_ie_i - \sum_{j \ne i} c_j e_j = E',
    \]
    so \(c_i'=0\) for every \(i=1,\dots,n\). Hence, \(E'\) is invariant under the reflections with respect to the coordinates planes, and therefore invariant under the map \(x\mapsto -x\) (the composition of all those reflections). Since for every unit vector \(v\) we can write \(E'=S_v(E)+c_v'v\), we have
    \begin{equation}\label{eq: gaining symmetry}
        S_v(E)+c'_vv =  -S_v(E) - c'_vv = S_v(E) + c'_v(-v).
    \end{equation}
    Clearly, \(c_{-v}'=-c_v'\) for every unit vector \(v\). On the other hand, combining \eqref{eq: gaining symmetry} with the fact that \(S_v(E)=S_{v'}(E)\), we also have that \(c_v'=c_{-v}'\). Therefore, for every unit vector \(v \in \R^n\), it must be \(c_v'=0\), that is, \(S_v(E')=E'\).
    
    As a consequence, \(E'\) is invariant under reflections by hyperplanes through the origin. Since by composition the set of those reflections generates all rotations that fix the origin, \(E'\) is invariant under any rotation fixing the origin. As a consequence, all the sections 
    \[
        E_v' \coloneq  \{ s \in \R \colon sv \in E'\}, \qquad v\in \R^n, \ |v|=1,
    \]
    must be equal, and since \(E'\) is bounded there exists \(r>0\) such that \(E_v'=(-r,r)\) for every unit vector \(v \in \R^n\). Hence, \(E'=B_r(0)\) and \(E=B_r(p)\), with \(p=(c_{e_1},\dots,c_{e_n})\).
\end{proof}

We can now prove the isoperimetric inequality.
\begin{proof}[Proof of Theorem~\ref{thm: isoperimetric inequality}]
    Suppose that \(E\) is bounded. Then taking \(R>0\) big enough, by Lemma~\ref{lemma: constrained isoperimetric inequality}, 
    \[
        P(E) \ge n \omega_n^{1/n} |E|^{\frac{n-1}{n}},
    \]
    and the equality holds if and only if \(E\) is equivalent to a ball.

    If \(E\) is not bounded, by Theorem~\ref{thm: polyhedral approximation} we can approximate it with bounded sets, that is, there are bounded sets \(E_h\subset \R^n\) with \(E_h \to E\) and \(P(E_h) \to P(E)\). Passing to the limit, the inequality is preserved. 
    
    Now assume that \(E=E^{(1)}\) is unbounded and \(|E|<\infty\). Assume by contradiction that 
    \[
        P(E) = n \omega_n^{1/n} |E|^{\frac{n-1}{n}}.
    \]
    By \eqref{thm: steiner inequality}, \(P(E)=P(S_v(E))\) for every unit vector \(v \in \R^n\). Up to a translation, arguing exactly as in Lemma~\ref{lemma: constrained isoperimetric inequality} we have that all the sections \(E\)
    \[
        E_v \coloneq  \{ s \in \R \colon sv \in E\}, \qquad v\in \R^n, \ |v|=1,
    \]
    are equal. Since \(E\) is unbounded, we have \(E_v=\R\) for every unit vector \(v \in \R^n\), but then \(E=\R^n\), a contradiction in because we assumed that \(|E|<\infty\).
\end{proof}


