Adopting the point of view of differential geometry, we could have used a completely different approach. Indeed, fix \(m>0\) and let
\begin{equation}\label{eq: isoperimetric problem}
    \alpha(n,m) \coloneq \inf\left\{ \Haus^{n-1}(\Sigma) \left| \ \begin{aligned}
        &\Sigma = \de \Omega \text{ smooth, connected,}\\
        &\text{compact hypersurface}, |\Omega|=m
    \end{aligned}
    \right.\right\}.
\end{equation}

Suppose that \(\Sigma= \de \Omega\) is a minimizer, that is, \(\Haus^{n-1}(\Sigma) = \alpha(m)\) and \(|\Omega|=m\). Let \(\nu \colon \Sigma \to \Sp^{n-1}\) be the outer unit normal of \(\Sigma\), \(A\) its second fundamental form, defined by 
\[
    A(X,Y)=\dif \nu_x [X] \cdot Y
\]
for every tangent vectors \(X,Y \in T_x \Sigma = T_{\nu(x)}\Sp^{n-1} \subset \R^n\), and \(H = \tr(A)\) its mean curvature. By the inverse function theorem and by the compactness of \(\Sigma\), there is \(\eps>0\) such that 
\[
    \phi(t,x) = x + t\nu(x),
\]
is a diffeomorphism on \((-\eps,\eps) \times \Sigma\). Let \(u\colon \Sigma \to \R\) be a smooth function with zero average, i.e., \(\int_\Sigma u \dif \Haus^{n-1}=0\), and define
\[
    F(t,x) = \phi(tu(x),x)
\]
for every \(x \in \Sigma\) and \(t \in \R\) small enough. Denote 
\[
    \Sigma_t = \{F(t,x) \colon x \in \Sigma\}.
\]
Then, \(\Sigma_t = \de \Omega_t\) is the boundary of a bounded set \(\Omega_t\). For every \(x \in \Sigma\), consider the geodesic
\[
    \gamma_x(s) = \phi(s,x),
\]
which has constant velocity \(|\dot \gamma_x|=u(x)\). Then, by the coarea formula,
\[
\begin{aligned}
    |\Omega_t| &= |\Omega| + |\{ \phi(s,x) \colon 0<s<tu(x)\}| - |\{ \phi(s,x) \colon tu(x)<s<0\}| \\
    &= m+ \int_{\{u>0\}} \int_0^t |\dot \gamma_x| \dif t  \dif \Haus^{n-1}(x) - \int_{\{u<0\}} \int_0^t |\dot \gamma_x| \dif t \dif \Haus^{n-1}(x)\\
    &= m + t\int_\Sigma u \ \dif \Haus^{n-1} = m,
\end{aligned}
\]
hence \(\Sigma_t\) is a competitor in \eqref{eq: isoperimetric problem} for every \(t\). By the minimality of \(\Sigma\),
\[
    \Haus^{n-1}(\Sigma_t) \ge \Haus^{n-1}(\Sigma),
\]
so the function \(t \mapsto \Haus^{n-1}(\Sigma_t)\) has a minimum at \(t=0\). The first variation formula implies that
\[
   0 = \left.\frac{\dif}{\dif t}\right|_{t=0} \Haus^{n-1}(\Sigma_t) = \int_\Sigma Hu \ \dif \Haus^{n-1}.
\]
Since this hold for every \(u:\Sigma \to \R\) with zero average, \(H\) is constant, and we say that \(\Sigma\) is \emph{CMC}.

Taking the second derivative with respect to \(t\), the second variation formula implies that
\begin{equation}\label{eq: stability inequality}
    0 \le \int_\Sigma |\nabla_\Sigma u|^2 - |A|^2 u^2  \ \dif \Haus^{n-1} = - \int_\Sigma u L_\Sigma u \ \dif \Haus^{n-1}
\end{equation}
for every smooth function \(u\) with zero average, where 
\[
    L_\Sigma u = \Delta_\Sigma u + |A|^2u
\]
is the \emph{Jacobi operator} of \(\Sigma\). Any hypersurface \(\Sigma\) that satisfies \eqref{eq: stability inequality} is called \emph{CMC-stable}. 

This approach can give another proof of the isoperimetric inequality. In particular, it holds the following stronger result.

\begin{thm}[Barbosa-do Carmo, {\cite{BarbosaDoCarmo}}]\label{thm: Barbosa-do Carmo}
    Let \(\Sigma\) be a compact, connected, orientable, immersed CMC hypersurface in \(\R^n\). Then \(\Sigma\) is CMC-stable if and only if \(\Sigma\) is a round sphere.
\end{thm}

The proof is basically just a computation: test \eqref{eq: stability inequality} with the function 
\[
    u(x) = H (\nu(x) \cdot x)-n+1 \qquad \forall x \in \Sigma.
\]

A.D. Alexandrov, with the celebrated \emph{moving plane method} (which heavily uses the strong maximum principle), proved that spheres are the only \emph{critical points} of the area functional with respect to variations with fixed volume.

\begin{thm}[Alexandrov, {\cite{Alexandrov1962}}]
    Let \(\Sigma\) be a compact, connected, embedded CMC hypersurface in \(\R^n\). Then \(\Sigma\) is a round sphere.
\end{thm}

\emph{A posteriori}, it is not restrictive to assume that \(\Sigma\) is connected in the isoperimetric problem \eqref{eq: isoperimetric problem}. In fact assume that \(\Sigma = \de \Omega\) is not connected. Let \(\Omega_1, \dots, \Omega_k\) be the connected components of \(\Omega\), and \(\Sigma_0^j,\dots,\Sigma_{N(j)}^j\) the connected components of \(\Sigma\) such that \(\de \Omega_j = \bigcup_i \Sigma_i^j\). Since each \(\Sigma_i^j\) is CMC-stable, by Theorem~\ref{thm: Barbosa-do Carmo} it's a round sphere, so \(\Sigma_i^j\) a boundary of a ball \(B_i^j\). Hence, every \(\Omega_j\) must be obtained by removing some disjoint balls \(B_1^j,\dots, B_{N(j)}^i\) from an initial big ball \(B_0^j\) of radius \(R_j\). Denote \(\Omega^* = \bigcup_j B_0^j \supset \Omega\) and \(\Sigma^* = \de \Omega^* \subset \Sigma\). Then it's easy to see (for example, by induction over \(k\)) that
\[
\begin{aligned}
    \Haus^{n-1}(\Sigma^*)^n &= \left(n \omega_n \sum_{j=1}^k R_j^{n-1}\right)^n = n^n \omega_n^n \left( \sum_{j=1}^k R_j^{n-1}\right)^n \\
    & \ge n^n \omega_n^n \left(\sum_{j=1}^k R_j^n\right)^{n-1} = n^n \omega_n^{n-1} \left( \omega_n \sum_{j=1}^k R_j^n\right)^{n-1}\\
    &= n^n \omega_n |\Omega^*|^{n-1}
\end{aligned}
\]
with strict inequality whenever \(k>1\), which is indeed the case if \(\Sigma=\Sigma^*\). If instead \(\Sigma \ne \Sigma^*\), then observe that
\[
    \Haus^{n-1}(\Sigma) > \Haus^{n-1}(\Sigma^*) \ge n \omega_n^{1/n}|\Omega^*|^{\frac{n-1}{n}} > n \omega_n^{1/n}|\Omega|^{\frac{n-1}{n}}.
\]


However, this approach encounters an insurmountable limit in high dimension: we are implicitly assuming that the category of sets with smooth boundary contains the minimizers of the isoperimetric problem. Although \emph{a posteriori} this is true, it is still is not obvious nor easy to prove \emph{a priori}. Indeed, for dimension \(n \ge 8\) there exist CMC-stable hypersurfaces that are the boundary of (unbounded) domains and that present singularities. This phenomena was first discovered by J. Simons in \cite{Simons_MinimalVarieties1968}, where he proved that the cone (which has been named after him)
\[
    C \coloneq \{ (x,y) \in \R^4 \times \R^4 \colon |x|=|y|\}
\]
is a stable minimal hypersurface away from 0, i.e., \(H=0\) and the inequality \eqref{eq: stability inequality} holds with every compactly supported function \(u\) on \(C\setminus\{0\}\). One year later, Bombieri, De Giorgi and Giusti proved that \(C\) is actually area minimizing \cite{BombieriDeGiorgiGiusti_Bernstein1969} and found similar examples for every \(n \ge 8\).

In general, an area minimizing ‘‘hypersurface'' \(\Sigma \subset \R^n\) (here, we should actually use the language of currents) must be smooth smooth if \(n \le 7\), while if \(n \ge 8\) \emph{a priori} we only know that its singular set has Hausdorff dimension at most \((n-1)-7\). This result has been proved with the hard work of many great mathematicians such as Fleming and De Giorgi (\(n=3\), \cite{Fleming, DeGiorgi1965}), Almgren (\(n=4\), \cite{Almgren_InteriorRegularity}), Simons (\(5\le n \le 7\), \cite{Simons_MinimalVarieties1968}) and Federer (\(n \ge 8\), \cite{Federer1970}). 