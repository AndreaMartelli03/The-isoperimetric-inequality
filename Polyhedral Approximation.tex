Given a function \(u \colon \R^{n-1} \to \R\) and a subset \(G \subset \R^{n-1}\), we define the \emph{graph of \(u\) over \(G\)} as
\[
    \Gamma(u,G) \coloneq \{ (x,u(x)) \in \R^n \colon x \in G\}.
\]

\begin{defn}
    A \emph{polyhedral set} is a finite, non empty intersection of closed half-spaces.
    
    A continuous function \(u \colon G \subset \R^n \to \R\) is said to be \emph{finitely piecewise affine} if there exists a finite partition of polyhedral sets \(\{G_i\}_{i=1}^k\) of \(G\) such that \(u\) is affine on each \(G_i\).

    A set \(E\subset \R^n\) is said to have \emph{polyhedral boundary} if for every \(x \in \de E\) there exist \(r>0\), a finitely piecewise affine function \(u \colon (-r,r)^{n-1} \to \R\) and an isometry \(T \colon \R^n \to \R^n\) with \(T(x)=0\) such that
    \[
    \begin{aligned}
        T(\de E) \cap (-r,r)^n &= G(u,(-r,r)^{n-1})\\
        T(E) \cap (-r,r)^n &= \{ (z,t) \in (-r,r)^{n-1}\times (-r,r) \colon t>u(z) \}.
    \end{aligned}
    \]
\end{defn}
\begin{rmk}
    It's not hard to prove that any function \(u \in C^1_c(\R^n)\) can be approximated by \emph{compactly supported finitely piecewise affine functions} in the \(C^1\)-norm.
\end{rmk}

\begin{lemma}[Approximation by finitely piecewise affine functions]\label{lemma: approximation by finitely piecewise affine functions}
    For every \(u \in BV(\R^n)\) there exists a sequence \(u_h \colon \R^n \to \R\) of compactly supported finitely piecewise affine functions such that \(u_h \to u\) in the BV-strict sense, that is, \(u_h \to u\) in \(L^1(\R^n)\) and
    \[
        \int_{\R^n} |\nabla u_h| \dif z \to |Du|(\R^n).
    \]
\end{lemma}
\begin{proof}
    \emph{Step 1}. Assume that \(u \in C^1(\R^n) \cap BV(\R^n)\). Take a sequence of cutoff functions \(\varphi_h \colon \R^n \to \R\), that is, for some sequence \(R_h \nearrow \infty\), smooth functions with compact support in \(B_{R_h+1/h}\), \(0 \le \vp \le 1\) on \(\R^n\), \(\varphi_h = 1\) on \(B_{R_h}\) and \(\|\nabla \vp_h\|_{C^0} \le C\), \(C>0\) not depending on \(h\). Then \(v_h \coloneq u\vp_h \in C^1_c(\R^n)\), \(v_h \to u\) in \(L^1(\R^n)\), and
    \[
    \begin{aligned}
        \int_{\R^n} |\nabla v_h| \dif x &= \int_{B_{R_h}}|\nabla u| \dif x + \int_{B_{R_h+1/h}\setminus B_{R_h}} |\nabla v_h| \dif x \\
        &\to \int_{\R^n}|\nabla u| \dif x \qquad \text{ as }h \to \infty.
    \end{aligned}
    \]
    For every \(h \in \N\), take a sequence \(v_h^k \colon \R^n \to \R\) of compactly supported finitely piecewise affine functions approximating each \(v_h\) in \(C^1_c(\R^n)\) with respect to the \(C^1\)-norm. Then in particular \(v_h^k \to v_h\) in \(L^1(\R^n)\) and 
    \[
        \lim_{k\to \infty}\int_{\R^n} |\nabla v_h^k| \dif x = \int_{\R^n}|\nabla v_h|\dif x.
    \]
    With a diagonal argument, we can extract a diagonal sequence \(u_h \coloneq v_h^{k(h)}\) that satisfies the required properties.
    
    \emph{Step 2}. Now take any \(u \in BV(\R^n)\), and fix a kernel mollifier \(\rho \in C^\infty_c(\R^n)\). Let \(v_h = u \star \rho_{\eps_h}\), with \(\eps_h \to 0\), and for each \(h \in \N\) take a sequence \(v_h^k \colon \R^n \to \R\) of finitely piecewise affine functions approximating each \(v_h\), i.e., such that \(v_h^k \to v_h\) in \(L^1(\R^n)\) and
    \[
        \lim_{k\to \infty}\int_{\R^n} |\nabla v_h^k| \dif x = \int_{\R^n}|\nabla v_h|\dif x.
    \]
    By the properties of mollification,
    \[
        \lim_{h \to \infty} \int_{\R^n} |\nabla v_h| \dif x = |Du|(\R^n),
    \]
    so, as before, we can extract a diagonal sequence \(u_h \coloneq v_h^{k(h)}\) that satisfies the required properties.
\end{proof}

\begin{thm}[Polyhedral approximation]\label{thm: polyhedral approximation}
    Let \(E \subset \R^n\) be a set with \(|E|<\infty\) and \(P(E)< \infty\). Then there exists a sequence of bounded sets \(E_h \subset \R^n\) with polyhedral boundary such that
    \[
        E_h \to E, \qquad P(E_h) \to P(E).
    \]
\end{thm}

%\begin{rmk}
%    Since \(E_h \to E\), we also have that \(D\chi_{E_h} \weakstarto D\chi_{E}\). Moreover, \(|D\chi_{E_h}|(\R^n) \to |D\chi_E|(\R^n)\), hence \(|D\chi_{E_h}|\weakstarto |D\chi_E|\).
%\end{rmk}

\begin{rmk}
In the proof, we are going to use the following formula: for every measurable set \(E\subset \R^n\) and \(u,v \in L^1_{loc}(\R^n)\), 
\begin{equation}\label{eq: L^1 norm as area}
    \int_E|u-v| \dif x = \int_\R \Leb^n(E \cap \{u>t\}\triangle \{v>t\}) \dif t,
\end{equation}
where \(A\triangle B = (A\setminus B) \cup (B \setminus A)\) is the symmetric difference. This is follows from the usual interpretation of \(\int_E |u-v|\) as the area between the graph of \(u\) and the graph of \(v\) over \(E\). Indeed, consider separately \(E^+=E \cap \{u>v\}\) and \(E^-=E \cap \{u<v\}\). By Fubini's theorem,
\begin{align*}
    \int_{E^+} |u-v| \dif x &= \int_{E^+} \Leb^1((v(x),u(x))) \dif x \\
    &= \Leb^{n+1}(\{(x,t) \in E\times\R \colon v(x)<t<u(x)\}) \\
    &= \int_\R \Leb^n(\{x \in E \colon v(x)<t<u(x)\}) \dif t \\
    &= \int_\R \Leb^n(E \cap \{u>t\}\setminus \{v>t\}) \dif t
\end{align*}
and similarly for \(E^-\).
\end{rmk}

%\begin{proof}
%    Without loss of generality let \(x=0\). The fact that \(E \cap B_R\) has finite perimeter follows from 
%    \[
%        P(E\cap B_R) = P(E\cap B_R;B_{R+1}) \le P(E;B_{R+1}) + P(B_R;B_{R+1}) < \infty.
%    \]
%    The rest of the proof can be found in \cite[Lemma~15.2]{Maggi_2012}.
%\end{proof}

\begin{proof}[Proof of Theorem~\ref{thm: polyhedral approximation}]
    By Lemma~\ref{lemma: approximation by finitely piecewise affine functions}, there is a sequence of compactly supported finitely piecewise affine functions \(u_h \colon \R^n \to \R\) such that \(u_h \to \chi_E\) in \(L^1(\R^n)\) and 
    \begin{equation}\label{eq: |Du_h|(R^n) -> P(E)}
        \lim_{h \to \infty} \int_{\R^n} |\nabla u_h| \dif x = P(E).
    \end{equation}

    For every \(h \in \N\) and \(t \in (0,1)\) consider the bounded set
    \[
        E_h^t \coloneq \{u_h >t\}.
    \]
    Since \(u_h\) is piecewise affine, \(\de E_h^t \subset \{u_h=t\}\) is always a polyhedra. We are going to choose a suitable \(t \in (0,1)\). 
    
    By \eqref{eq: L^1 norm as area} we have that 
    \[
    \begin{aligned}
        \int_0^1 \int_{\R^n}|\chi_{E^t_h}-\chi_{E}| \dif x \dif t & = \int_0^1 \int_{\R^n}|\chi_{\{u_h>t\}}- \chi_{\{\chi_E>t\}}| \dif x \dif t \\
        &= \int_0^1 \Leb^n(\{u_h >t\} \triangle \{\chi_E>t\}) \dif t\\
        &\le \int_\R \Leb^n(B \cap \{u_h >t\} \triangle \{\chi_E>t\}) \dif t\\
        &= \int_{\R^n} |u_h-\chi_E| \dif x \to 0,
    \end{aligned}
    \]
    so \(E_h^t \to E\) for a.e. \(t \in (0,1)\), and in particular
    \begin{equation}\label{eq: P(E) < liminf_h P(E_h^t) a.e. t}
        P(E) \le \liminf_{h \to \infty} P(E_h^t) \text{ for a.e. }t \in (0,1)
    \end{equation}
    by the lower semicontinuity of the perimeter.
    By \eqref{eq: |Du_h|(R^n) -> P(E)},
    \[
    \begin{aligned}
        P(E) &= \lim_{h \to \infty}\int_{\R^n} |\nabla u_h| \dif x &&\\
        &= \lim_{h \to \infty} \int_\R P(\{u_h >t\}) \dif t &&\text{(coarea formula)}\\
        & \ge \int_\R \liminf_{h \to \infty} P(\{u_h>t\}) \dif t &&\text{(Fatou's lemma)}\\
        & \ge \int_0^1 \liminf_{h \to \infty} P(E_h^t) \dif t &&\\
        &\ge P(E),
    \end{aligned}
    \]
    having used \eqref{eq: P(E) < liminf_h P(E_h^t) a.e. t} in the last inequality. Therefore,
    \[
        P(E)= \int_0^1 \liminf_{h \to \infty} P(E_h^t) \dif t,
    \]
    and again by \eqref{eq: P(E) < liminf_h P(E_h^t) a.e. t} it must be
    \begin{equation*}
        P(E) = \liminf_{h \to \infty}P(E_h^t) \qquad \text{ for a.e. }t \in (0,1).
    \end{equation*}
    Up to passing to a subsequence (which we don't relabel), 
    \begin{equation}\label{eq: P(E)=lim_hP(E_h^t;A)}
        P(E) = \lim_{h \to \infty}P(E_h^t) \qquad \text{ for a.e. }t \in (0,1).
    \end{equation}
    Thus we can find \(t \in (0,1)\) such that \(E_h\coloneq E_h^t\) is as required.
\end{proof}